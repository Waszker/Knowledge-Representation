\documentclass{article}
\usepackage[polish]{babel}
\usepackage[utf8]{inputenc}
\usepackage{polski}
\frenchspacing
\setcounter{tocdepth}{2}

\begin{document}
	
\begin{titlepage}

\newcommand{\HRule}{\rule{\linewidth}{0.5mm}}

\center

%----------------------------------------------------------------------------------------

\textsc{\LARGE Politechnika Warszawska}\\[5mm]
\textsc{\LARGE Wydział Matematyki i Nauk Informacyjnych}\\[3.5cm]
 
%----------------------------------------------------------------------------------------

\textsc{\Huge Reprezentacja wiedzy}\\[0.5cm]

%----------------------------------------------------------------------------------------

\HRule \\[0.4cm]
{ \LARGE \bfseries Programy działań z efektami domyślnymi}\\[2.5cm]
 
%----------------------------------------------------------------------------------------

\begin{flushright}
\Large \emph{Autorzy:}\\[0.5cm]
Dragan Łukasz\\
Flis Mateusz\\
Fusiara Marcin\\
Izert Piotr\\
Pielat Mateusz\\
Rząd Przemysław\\
Siry Roman\\
Waszkiewicz Piotr\\
Zawadzka Anna\\

\end{flushright}
%----------------------------------------------------------------------------------------

\vfill
{\large \today}\\[3cm]

\end{titlepage}
	
\newpage

\section{Opis zadania}

Zadaniem projektu jest opracowanie i zaimplementowanie języka akcji dla specyfikacji podanej klasy systemów dynamicznych oraz odpowiadający mu język kwerend.\\

System dynamiczny spełnia podane założenia:
\begin{enumerate}
\item Prawo inercji
\item Niedeterminizm i sekwencyjność działań
\item Pełna informacja o wszystkich akcjach i wszystkich ich skutkach bezpośrednich
\item Z każdą akcją związany jest:
\begin{enumerate}
\item Warunek początkowy (ew. true)
\item Efekt akcji
\item Jej wykonawca
\end{enumerate}
\item Skutki akcji:
\begin{enumerate}
\item Pewne (zawsze występują po zakończeniu akcji)
\item Domyślne (preferowane. Zachodzą po zakończeniu akcji, o ile nie jest wiadomym, że nie występują)
\end{enumerate}
\item Efekty akcji zależą od jej stanu, w którym akcja się zaczyna i wykonawcy tej akcji
\item W pewnych stanach akcje mogą być niewykonalne przez pewnych (wszystkich) wykonawców
\end{enumerate}

Programem działań nazywać będziemy ciąg $((A_{1},w_{1}), (A_{2},w_{2}), …, (A_{n},w_{n}))$, 
gdzie $A_{i}$ jest akcją, zaś $w_{i}$ jej wykonawcą lub $\epsilon$ (ktokolwiek).\\


Język kwerend zapewnia uzyskanie odpowiedzi na następujące pytania:
\begin{enumerate}
\item Czy podany program działań jest wykonywalny zawsze/kiedykolwiek?
\item Czy wykonanie podanego programu działań z dowolnego stanu spełniającego warunek $\pi$ prowadzi zawsze/kiedykolwiek/na ogół do stanu spełniającego warunek celu $\gamma$ ?
\item Czy z dowolnego stanu spełniającego warunek $\pi$ cel $\gamma$ jest osiągalny zawsze/kiedykolwiek/na ogół?
\item Czy wskazany wykonawca jest zaangażowany w realizację programu zawsze/kiedykolwiek?
\end{enumerate}


\section{Język akcji $\Omega$}

\subsection{Definicja języka}
$\Omega$ jest rodziną języków, w której każdy język $\mathcal{L}$ określony jest nad sygnaturą 
\begin{center}
$\Upsilon=(F,A,W)$
\end{center}
gdzie: 
\begin{itemize}
\item $F$ - niepusty zbiór zmiennych (fluenty)
\item $A$ - niepusty zbiór akcji
\item $W$ - niepusty zbiór wykonawców (aktorów), przy czym $\epsilon\in W$, gdzie $\epsilon$ oznacza kogokolwiek
\end{itemize}

\subsection{Syntaktyka języka} 
W języku  $\Omega$ występują następnujące typy zdań:
\begin{itemize}
\item {\large\texttt{initially $\alpha$}}\\
formuła $\alpha$ zachodzi w stanie początkowym
\item {\large\texttt{$\alpha$ after $(A_{1},w_{1}), ..., (A_{n},w_{n})$}}\\
formuła $\alpha$ zachodzi po wykonaniu sekwencji $(A_{1},w_{1}), ..., (A_{n},w_{n})$, gdzie $A_{i}$ jest akcją, zaś $w_{i}$ jej wykonawcą
\item {\large\texttt{$(A,w)$ causes $\alpha$}}\\
skutkiem wykonania akcji $A$ przez wykonawcę $w$ jest stan, w którym spełniona jest formuła $\alpha$
\item {\large\texttt{$(A,w)$ causes $\alpha$ if $\pi$}}\\
skutkiem wykonania akcji $A$ przez wykonawcę $w$ w stanie spełniającym warunek $\pi$ jest stan, w którym spełniona jest formuła $\alpha$
\item {\large\texttt{observable $\alpha$ after $(A_{1},w_{1}), ..., (A_{n},w_{n})$}}\\
po wykonaniu sekwencji $(A_{1},w_{1}), ..., (A_{n},w_{n})$, gdzie $A_{i}$ jest akcją, zaś $w_{i}$ jej wykonawcą, w stanie początkowym może (ale nie musi) zachodzić formuła $\alpha$
\item {\large\texttt{impossible $(A,w)$ if $\pi$}}\\
niemożliwe jest wykonanie akcji $A$ przez wykonawcę $w$ w stanie spełniającym warunek $\pi$
\item {\large\texttt{$(A,w)$ releases $f$ if $\pi$}}\\
wykonanie akcji $A$ przez wykonawcę $w$ w stanie spełniającym warunek $\pi$ może (ale nie musi) zmienić wartość zmiennej $f$
\item {\large\texttt{$(A,w)$ typically causes $\alpha$ if $\pi$}}\\
skutkiem wykonania akcji $A$ przez wykonawcę $w$ w stanie spełniającym warunek $\pi$ na ogół jest stan, w którym spełniona jest formuła $\alpha$
\item {\large\texttt{always $\alpha$}}\\
formuła $\alpha$ jest spełniona w każdym stanie
\end{itemize}
gdzie $\alpha$ jest dowolną kombinacją zmiennych (fluentów): 
\begin{center}
$\alpha= f | \alpha | \neg\alpha | \alpha_{1} \land \alpha_{2} | \alpha_{1} \lor \alpha_{2} | \alpha_{1} \to \alpha_{2} | \alpha_{1} \leftrightarrow \alpha_{2} $
\end{center}


\subsection{Semantyka języka} 

\subsubsection{Stan}
Stanem będziemy nazywać dowolną fukcję $\sigma:F\to \{1,0\}$, która przypisuje zmiennym wartości logiczne. Jeśli $\sigma(f)=1$, to znaczy, że zmienna $f$ zachodzi w stanie $\sigma$. Funkcję tę można rozszerzyć na zbiór wszystkich formuł nad zbiorem zmiennych $F$ wedug zasad obowiązujących w klasycznej logice zdań.

\subsubsection{Struktura}
Strukturą nazywamy układ $S=(\Sigma, \sigma_{0}, ResAb, ResN)$, gdzie:
\begin{itemize}
\item $\Sigma$ - zbiór stanów
\item $\sigma_{0} \in \Sigma$ - stan początkowy
\item $ResAb, ResN$ : $A\times W \times \Sigma \to 2^{\Sigma}$ są funkcjami przejść. $ResAb$ jest funkcją przejść nietypowych, $ResN$ jest funkcją przejść typowych oraz $ResAb \cap ResN= \emptyset$ 
\end{itemize}

\subsubsection{Model dziedziny}
W celu zdefiniowania pojęcia modelu dziedziny wprowadzone zostaną następujące funkcje pomocnicze:
\begin{enumerate}
	\item $Res_{0}$ : $A \times W \times \Sigma \to 2^{\Sigma}$ konstruowane na podstawie zdań efektów akcji.
	\[ \forall_{a \in A, w \in W, \sigma \in \Sigma} Res_{0}(a,w,\sigma) = \{\sigma' \in \Sigma: ((a, w) \textbf{ causes } \alpha \textbf{ if } \pi) \in D \land (\sigma \models \pi) \Rightarrow (\sigma' \models \alpha) \}  \]
	Oznacza to, że $Res_{0}$ konstruuje się bez minimalizacji zmian.
	\item Funkcję $Res^{-}$ wyznacza się stosując minimalizację zmian.
	\item Funkcję $Res^{+} : A \times W \times \Sigma \to 2^{\Sigma}$ spełniającą warunek $\forall_{a \in A, w \in W, \sigma \in \Sigma}$:
	\[ Res_{0}^{+}(a, w,\sigma) = \]
	\[ \{\sigma' \in Res_{0}(a, w,\sigma) : ((a, w) \textbf{ typically causes } \beta \textbf{ if } \pi) \in D \land (\sigma \models \varphi) \Rightarrow (\sigma' \models \beta) \}  \]
\end{enumerate}

Niech D będzie dziedziną akcji języka $\Omega$ i niech $S=(\Sigma, \sigma_{0}, ResAb, ResN)$ będzie strukturą dla $\Omega$. Mówimy, że S jest modelem D $\leftrightarrow$ spełnione są warunki:
\begin{itemize}
	\item $\Sigma$ jest zbiorem stanów z dziedziny D
	\item każde zdanie obserwacji i każde zdanie wartości z dziedziny D jest prawdziwe w S
	\item $\forall_{ a \in A, w \in W, \sigma \in \Sigma } ResN(a, w, \sigma)$ jest zbiorem tych wszystkich stanów $\sigma' \in Res_{0}^{+}(a, w, \sigma)$, dla których zbiory $New(a, w, \sigma, \sigma')$ są minimalne
	\item $\forall_{a \in A, w \in W, \sigma \in \Sigma} ResAb(a, w, \sigma) = Res^{-}(a, w, \sigma) | ResN(a, w, \sigma)$
\end{itemize}
Warto zwrócić uwagę na to, że skutki \textit{pewne} dla akcji traktowane są jak \textit{typowe}.

\subsubsection{Funkcja częściowa}
Niech $S=(\Sigma,\sigma_{0},ResAb,ResN)$ będzie strukturą dla języka. Konstrukcja funkcji $\Psi_{S} : (A \times W)^{*} \times \Sigma \to \Sigma$ wygląda następująco:
\begin{itemize}
	\item $\Phi_{S}(a,\epsilon,\sigma)=\sigma$ gdzie $\epsilon$ oznacza ciąg pusty
	\item jeśli $\Phi_{S}(((a_{1}, w_{1}), \dots, (a_{n}, w_{n})),\sigma)$ jest określona to
	\[\Phi_{S}(((a_{1}, w_{1}), \dots, (a_{n}, w_{n})),\sigma) \in ResAb((a_{n}, w_{n}), \Phi_{S}((a_{1}, w_{1}),\dots,(a_{n-1}, w_{n-1}))) \]
	\[ \cup ResN((a_{n}, w_{n}), \Phi_{S}((a_{1}, w_{1}),\dots,(a_{n-1}, w_{n-1})))\]
\end{itemize}


\section{Język kwerend}
\begin{itemize}
\item \textbf{Czy podany program działań jest wykonywalny zawsze/kiedykolwiek?}\\ 
always/ever executable $SC$
\item \textbf{Czy wykonanie podanego programu działań z dowolnego stanu spełniającego warunek $\pi$ prowadzi zawsze/kiedykolwiek/na ogół do stanu spełniającego warunek celu $\gamma$ ?}\\ 
always/ever/typically accessible $\gamma$ if $\pi$ when $SC$
\item \textbf{Czy z dowolnego stanu spełniającego warunek $\pi$ cel $\gamma$ jest osiągalny zawsze/kiedykolwiek/na ogół?}\\ 
always/ever/typically accessible $\gamma$ if $\pi$
\item \textbf{Czy wskazany wykonawca jest zaangażowany w realizację programu zawsze/kiedykolwiek?}\\ always/ever partakes $w$ when $SC$

\end{itemize}


\end{document}
