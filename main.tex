\documentclass{article}
\usepackage[polish]{babel}
\usepackage[utf8]{inputenc}
\usepackage{polski}
\frenchspacing
\setcounter{tocdepth}{2}

\begin{document}
	
\begin{titlepage}

\newcommand{\HRule}{\rule{\linewidth}{0.5mm}}

\center

%----------------------------------------------------------------------------------------

\textsc{\LARGE Politechnika Warszawska}\\[5mm]
\textsc{\LARGE Wydział Matematyki i Nauk Informacyjnych}\\[3.5cm]
 
%----------------------------------------------------------------------------------------

\textsc{\Huge Reprezentacja wiedzy}\\[0.5cm]

%----------------------------------------------------------------------------------------

\HRule \\[0.4cm]
{ \LARGE \bfseries Programy działań z efektami domyślnymi}\\[2.5cm]
 
%----------------------------------------------------------------------------------------

\begin{flushright}
\Large \emph{Autorzy:}\\[0.5cm]
Dragan Łukasz\\
Flis Mateusz\\
Fusiara Marcin\\
Izert Piotr\\
Pielat Mateusz\\
Rząd Przemysław\\
Siry Roman\\
Waszkiewicz Piotr\\
Zawadzka Anna\\

\end{flushright}
\\[1.2cm]
%----------------------------------------------------------------------------------------

\vfill
{\large \today}\\[3cm]

\end{titlepage}
	
\newpage

\section{Opis zadania}

Zadaniem projektu jest opracowanie i zaimplementowanie języka akcji dla specyfikacji podanej klasy systemów dynamicznych oraz odpowiadający mu język kwerend.\\

System dynamiczny spełnia podane założenia:
\begin{enumerate}
\item Prawo inercji
\item Niedeterminizm i sekwencyjność działań
\item Pełna informacja o wszystkich akcjach i wszystkich ich skutkach bezpośrednich
\item Z każdą akcją związany jest:
\begin{enumerate}
\item Warunek początkowy (ew. true)
\item Efekt akcji
\item Jej wykonawca
\end{enumerate}
\item Skutki akcji:
\begin{enumerate}
\item Pewne (zawsze występują po zakończeniu akcji)
\item Domyślne (preferowane. Zachodzą po zakończeniu akcji, o ile nie jest wiadomym, że nie występują)
\end{enumerate}
\item Efekty akcji zależą od jej stanu, w którym akcja się zaczyna i wykonawcy tej akcji
\item W pewnych stanach akcje mogą być niewykonalne przez pewnych (wszystkich) wykonawców
\end{enumerate}

Programem działań nazywać będziemy ciąg $((A_{1},w_{1}), (A_{2},w_{2}), …, (A_{n},w_{n}))$, 
gdzie $A_{i}$ jest akcją, zaś $w_{i}$ jej wykonawcą lub $\epsilon$ (ktokolwiek).\\


Język kwerend zapewnia uzyskanie odpowiedzi na następujące pytania:
\begin{enumerate}
\item Czy podany program działań jest wykonywalny zawsze/kiedykolwiek?
\item Czy wykonanie podanego programu działań z dowolnego stanu spełniającego warunek $\pi$ prowadzi zawsze/kiedykolwiek/na ogół do stanu spełniającego warunek celu $\gamma$ ?
\item Czy z dowolnego stanu spełniającego warunek $\pi$ cel $\gamma$ jest osiągalny zawsze/kiedykolwiek/na ogół?
\item Czy wskazany wykonawca jest zaangażowany w realizację programu zawsze/kiedykolwiek?
\end{enumerate}


\section{Język akcji}

Język definiujemy jako parę:
$\Omega = (F,Ac)$, gdzie: F - niepusty zbiór zmiennych (fluenty), Ac - niepusty zbiór akcji. Akcję definiujemy jako parę (A,w), gdzie A - akcja, w - wykonawca.\\

\textbf{Syntaktyka języka} 
\begin{itemize}
\item initially $\alpha$
\item $\alpha$ after $(A_{1},w_{1}), ..., (A_{n},w_{n})$
\item $(A,w)$ causes $\alpha$
\item $(A,w)$ causes $\alpha$ if $\pi$
\item observable $\alpha$ after $(A_{1},w_{1}), ..., (A_{n},w_{n})$
\item impossible $(A,w)$ if $\pi$
\item $(A,w)$ releases $f$ if $\pi$
\item $(A,w)$ typically causes $\alpha$ if $\pi$
\item always $\alpha$
\end{itemize}


\section{Język kwerend}
\begin{itemize}
\item \textbf{Czy podany program działań jest wykonywalny zawsze/kiedykolwiek?}\\ 
always/ever executable $SC$
\item \textbf{Czy wykonanie podanego programu działań z dowolnego stanu spełniającego warunek $\pi$ prowadzi zawsze/kiedykolwiek/na ogół do stanu spełniającego warunek celu $\gamma$ ?}\\ 
always/ever/typically accessible $\gamma$ if $\pi$ when $SC$
\item \textbf{Czy z dowolnego stanu spełniającego warunek $\pi$ cel $\gamma$ jest osiągalny zawsze/kiedykolwiek/na ogół?}\\ 
always/ever/typically accessible $\gamma$ if $\pi$
\item \textbf{Czy wskazany wykonawca jest zaangażowany w realizację programu zawsze/kiedykolwiek?}\\ always/ever partakes $w$ when $SC$

\end{itemize}


\end{document}
