\documentclass{article}
\usepackage{float}
\usepackage[polish]{babel}
\usepackage[utf8]{inputenc}
\usepackage{polski}
\usepackage{algpseudocode}
\usepackage{algorithm}
\usepackage{hyperref}
\frenchspacing
\setcounter{tocdepth}{2}
\usepackage{graphicx}
\graphicspath{ {images/} }

%%% fix for \lll
\let\babellll\lll
\let\lll\relax

\usepackage{amssymb}

%%% fix for \lll
\let\mathlll\lll
\let\lll\babellll


\begin{document}
\begin{titlepage}
\newcommand{\HRule}{\rule{\linewidth}{0.5mm}}
\newcommand{\Action}[1]{\textsc{#1}}
\center
%----------------------------------------------------------------------------------------
\textsc{\LARGE Politechnika Warszawska}\\[0.3cm]
\textsc{\Large Wydział Matematyki i Nauk Informacyjnych}\\[0.6cm]
% logo
\includegraphics[width=2cm, height=2cm]{logo}\\[0.6cm]
\textsc{\Huge Reprezentacja wiedzy}\\[0.3cm]
%---------------------------------------------------------------------------------------
\HRule \\[0.4cm]
{ \LARGE \bfseries Programy działań z efektami domyślnymi}\\[0.1cm]
 
%----------------------------------------------------------------------------------------
\HRule \\[0.4cm]
{  \bfseries Komentarz do raportu z testów}\\[1.2cm]
% authors
\begin{flushright}
\Large \emph{Autorzy:}\\[0.5cm]
Dragan Łukasz\\
Flis Mateusz\\
Izert Piotr\\
Pielat Mateusz\\
Rząd Przemysław\\
Siry Roman\\
\textbf{Waszkiewicz Piotr}\\
Zawadzka Anna\\[0.9cm]
\end{flushright}
% date
\vfill
{\large 14 czerwca 2016}\\[1cm]
\end{titlepage}
\newpage


Nasz projekt został przetestowany przez zespół nr 2 (Rafał Szczekutek - szef zespołu, Mateusz Bukowski  Piotr Chmiel, Martyna Grotek, Łukasz Napora, Kamil Żak). \\
Poniżej znajdują się komentarze odnoszące się do raportu z przeprowadzonych testów.
\section{Testy uruchamialności programu}
\subsection{Okno 1 - definiowanie sygnatury języka}
\begin{itemize}
        \item Punkt 4. \textit{Brak funkcjonalności usunięcia wielu  fluentów/akcji/wykonawców jednocześnie}.\\
        Funkcjonalność ta nie została przewidziana w projekcie. Nie jest ona niezbędna do prawidłowego działania programu.
    \end{itemize}
    
\subsection{Okno 2 - definiowanie dziedziny języka}
\begin{itemize}
        \item Punkt 3. \textit{Brak funkcjonalności usunięcia jednocześnie wielu zdań z języka}.\\
        Funkcjonalność ta nie została przewidziana w projekcie. Nie jest ona niezbędna do prawidłowego działania programu.
        \item Punkt 4. \textit{Nieobsłużony wyjątek}.\\
        Rzeczywiście przy definiowaniu dziedziny stan początkowy powinien być tylko jeden, natomiast zdefniowana formuła skutkuje utworzeniem więcej niż jednego stanu początkowego, co spowodowało wyjątek i błąd aplikacji. Problem został obsłużony, a w wypadku zdefiniowania takiej formuły początkowej, która zakłada istnienie wielu stanów początkowych, wyświetlany jest odpowiedni komunikat.
        \item Punkt 5. \textit{Brak aktualizacji okna definiowania dziedziny}.\\
        Błąd został poprawiony, okno definiowania dziedziny jest odświeżane przy każdym przejściu z okna definiowania sygnatury języka.
        \item Punkt 7. \textit{Błąd krytyczny przy dodaniu dwóch identycznych zdań \texttt{initially}}.\\
        Błąd poprawiono.
        \item Punkt 8. \textit{Nieobsłużony wyjątek}.\\
        Błąd identyczny jak w punkcie 4, naprawiono.
    \end{itemize}


\end{document}