\documentclass{article}
\usepackage{float}
\usepackage[polish]{babel}
\usepackage[utf8]{inputenc}
\usepackage{polski}
\usepackage{listings} %code environment
\frenchspacing
\setcounter{tocdepth}{2}
\usepackage{graphicx}
\graphicspath{ {images/} }
\usepackage{color}

\lstdefinestyle{main}{
	keywords={
		after, always, causes, if, impossible, initially, observable, 
		partakes, releases, typically
    },
    keywordstyle={\bfseries},
    keywordstyle=[2]\textsc,
    frame=single,
}

\begin{document}
	
\begin{titlepage}

\newcommand{\HRule}{\rule{\linewidth}{0.5mm}}
\newcommand{\Action}[1]{\textsc{#1}}

\center

%----------------------------------------------------------------------------------------

\textsc{\LARGE Politechnika Warszawska}\\[0.3cm]
\textsc{\Large Wydział Matematyki i Nauk Informacyjnych}\\[0.6cm]

% logo
\includegraphics[width=2cm, height=2cm]{logo}\\[0.6cm]


\textsc{\Huge Reprezentacja wiedzy}\\[0.3cm]

%----------------------------------------------------------------------------------------

\HRule \\[0.4cm]
{ \LARGE \bfseries Raport z testów projektu grupy nr 4}\\[0.1cm]
 
%----------------------------------------------------------------------------------------

\HRule \\[0.4cm]
{  \bfseries Programy działań z akcjami współbieżnymi}\\[1.2cm]

% authors
\begin{flushright}
\Large \emph{Autorzy:}\\[0.5cm]
Dragan Łukasz\\
Flis Mateusz\\
Izert Piotr\\
Pielat Mateusz\\
Rząd Przemysław\\
Siry Roman\\
\textbf{Waszkiewicz Piotr}\\
Zawadzka Anna\\[0.9cm]

\end{flushright}

% date
\vfill
{\large \today}\\[1cm]
	


\end{titlepage}
	
\newpage

\section{Opis projektu}
Tematem testowanego przez nas projektu są programy działań z akcjami współbieznymi. Rozpatrywana klasa systemów dynamicznych spełnia następujące warunki:
\begin{itemize}
    \item Prawo inercji
    \item Niedeterminizm
    \item W języku kwerend występują akcje złożone (zbiory co najwyzej k akcji atomowych), w języku akcji jedynie akcje atomowe
    \item Pełna informacja o wszystkich akcjach atomowych i wszystkich ich skutkach bezpośrednich
    \item Z każdą akcją atomową zwiazany jest jej warunek wstepny (ew. TRUE) i końcowy (efekt akcji)
    \item Wykonywane są jedynie akcje bezkonfliktowe (żadne dwie akcje składowe nie mogą mieć wspólnych zmiennych, na które w jakimkolwiek stanie mają wpływ)
    \item Wynikiem akcji złożonej jest suma skutków wszystkich składowych akcji bezkonfliktowych
    \item Akcje mogą być niewykonalne w pewnych stanach; jeśli akcja jest niewykonalna, to każda akcja ją zawierająca jest niewykonalna
    \item Dopuszczalny jest opis częściowy zarówno stanu poczatkowego, jak i pewnych stanów wynikajacych z wykonań sekwencji akcji
\end{itemize}

Opracowywany jezyk kwerend ma za zadanie umożliwiać tworzenie zapytań,
pozwalających na uzyskanie odpowiedzi na następujace pytania:
\begin{itemize}
    \item Czy podany program P działań jest możliwy do realizacji  zawsze/kiedykolwiek ze stanu początkowego?
    \item Czy wykonanie programu P działań w stanie początkowym prowadzi zawsze/kiedykolwiek do osiągniecia celu $\gamma$?
    \item Czy cel $\gamma$ jest osiagalny ze stanu początkowego?
\end{itemize}
\newpage

\section{Przeprowadzone testy}
\subsection{Test 1}
Zdefiniowana dziedzina:
\bigskip
\lstset{
	style=main,
	keywords=[2]{Load, Shoot},
}
\begin{lstlisting}[mathescape=true]
initially alive
initially $\neg$loaded
Load causes loaded
Shoot causes $\neg$loaded if loaded
Shoot causes $\neg$alive if loaded
\end{lstlisting}
\vspace{1cm}
Zadane kwerendy:
\begin{itemize}
    \item {\large\texttt{ever accessible $\neg$alive}}\\
    Odpowiedź: \texttt{TRUE}\\
    Jest to odpowiedź poprawna, ponieważ istnieje ciąg akcji, który prowadzi do stanu spełniającego podany cel, co ilustruje poniższy graf:
    \begin{figure}[H]
    \centering
    \includegraphics[scale=0.5]{test1_1}
    \end{figure}
    \item {\large\texttt{always accessible $\neg$alive}}\\
    Odpowiedź: \texttt{FALSE}\\
    Jest to prawda, ponieważ nie wszystkie ciągi akcji prowadzą do stanu, który spełnia podany warunek.
    \begin{figure}[H]
    \centering
    \includegraphics[scale=0.5]{test1_2}
    \end{figure}
    \item {\large\texttt{ever $\neg$alive after SHOOT, LOAD}}\\
    Odpowiedź: \texttt{FALSE}\\
    Jest to odpowiedź poprawna, ponieważ ze stanu początkowego, w którym zmienna $loaded$ nie jest prawdziwa, akcje $SHOOT$ i $LOAD$ nie zmienią stanu zmiennej $alive$.
    \begin{figure}[H]
    \centering
    \includegraphics[scale=0.5]{test1_3}
    \end{figure}
    \item {\large\texttt{always executable SHOOT, LOAD, SHOOT}}\\
    Odpowiedź: \texttt{TRUE}\\
    Odpowiedź jest zgodna z prawdą - w dziedzinie nie istnieją zadania, które miałyby uniemożliwiać wykoanie ciągu podanych akcji.
    \begin{figure}[H]
    \centering
    \includegraphics[scale=0.5]{test1_4}
    \end{figure}
\end{itemize}

\newpage
\subsection{Test 2}
Zdefiniowana dziedzina:
\bigskip
\lstset{
	style=main,
	keywords=[2]{Eat, Shop, Cook},
}
\begin{lstlisting}[mathescape=true]
initially hungry $\wedge$ angry
initially emptyFridge $\wedge$ $\neg$hasMeal $\wedge$ $\neg$chaos 
impossible Eat if $\neg$hasMeal 
impossible Cook if emptyFridge
Shop causes $\neg$emptyFridge
Shop releases angry if angry
Cook causes chaos if angry
Cook causes hasMeal 
Cook releases emptyFridge if emptyFridge
Eat causes $\neg$hasMeal $\wedge$ $\neg$hungry $\wedge$ $\neg$angry
\end{lstlisting}
\vspace{1cm}
Zadane kwerendy:
\begin{itemize}
    \item {\large\texttt{always executable COOK, SHOP}}\\
    Odpowiedź: \texttt{FALSE}\\
    Odpowiedż jest poprawna, ponieważ zdanie impossible uniemożliwia wykonanie akcji $COOK$ w przypadku gdy zmienna $emptyFridge$ jest prawdziwa.
    \begin{figure}[H]
    \centering
    \includegraphics[scale=0.5]{test2_1}
    \end{figure}
    \item {\large\texttt{always executable \{COOK, SHOP\}}}\\
    Odpowiedź: \texttt{TRUE}\\
    Odpowiedż jest poprawna, ponieważ w tym wypadku zdefiniowana została akcja złożona z akcji $COOK$ i $SHOP$. Akcja $COOK$ nie jest wykonywalna ze stanu początkowego, więc pod uwagę brana jest tylko akcja $SHOP$, która jestw wykonywalna ze stanu początkowego. 
    \begin{figure}[H]
    \centering
    \includegraphics[scale=0.5]{test2_2}
    \end{figure}
    \item {\large\texttt{ever accessible emptyFridge $\wedge$ $\neg$angry $\wedge$ hungry $\wedge$ $\neg$chaos }}\\
    Odpowiedź: \texttt{TRUE}\\
    Jest to odpowiedź poprawna, jednak nie jest zrozumiałe, dlaczego w grafie zwróconym przez program widnieją stany, które nie spełniają podanych założeń i nie należą do ścieżki, która prowadzi do takiego stanu.
    \begin{figure}[H]
    \centering
    \includegraphics[scale=0.35]{test2_3}
    \end{figure}
    \item {\large\texttt{ever $\neg$emptyFridge after SHOP, COOK}}\\
    Odpowiedź: \texttt{TRUE}\\
    Odpowiedź jest zgodna z prawdą, ponieważ akcja $COOK$ może, ale nie musi zmienić stan zmiennej $emptyFridge$.
    \begin{figure}[H]
    \centering
    \includegraphics[scale=0.35]{test2_4}
    \end{figure}
\end{itemize}

\newpage
\subsection{Test 3 - kapelusznik w krainie czarów}
Zdefiniowana dziedzina:
\bigskip
\lstset{
	style=main,
	keywords=[2]{Eat, Drink},
}
\begin{lstlisting}[mathescape=true]
initially $\neg$hatterMad $\wedge$ cakeExists $\wedge$ elixirExists
drink causes hatterMad if elixirExists
eat causes $\neg$hatterMad
impossible eat if $\neg$cakeExists
drink releases elixirExists if elixirExists
eat causes $\neg$cakeExists
\end{lstlisting}
\vspace{1cm}
Zadane kwerendy:
\begin{itemize}
    \item {\large\texttt{ever cakeExists after eat}}\\
    Oczekiwana odpowiedź: \texttt{FALSE}\\
    Odpowiedź programu: \texttt{FALSE}\\
    Odpowiedż jest poprawna, ponieważ jedzenie zawsze powoduje brak ciastka, a w dziedzinie nie ma możliwości przywrócenia ciastka.
    \item {\large\texttt{always executable drink, drink}}\\
    Oczekiwana odpowiedź: \texttt{TRUE}\\
    Odpowiedź programu: \texttt{TRUE}\\
    Odpowiedż jest poprawna, ponieważ nawet jeśli eliksir się skończy, akcja picia jest wykonywalna.
    \item {\large\texttt{always accessible hatterMad}}\\
    Oczekiwana odpowiedź: \texttt{TRUE}\\
    Odpowiedź programu: \texttt{TRUE}\\
    Odpowiedź jest poprawna.
    \item {\large\texttt{ever ~hatterMad after drink, eat, drink, eat}}\\
    Oczekiwana odpowiedź: \texttt{FALSE}\\
    Odpowiedź programu: \texttt{FALSE}\\
    Odpowiedż jest poprawna - druga próba zjedzenia ciastka jest niewykonalna.
    \newpage
    \item {\large\texttt{ever $\neg$hatterMad after drink, eat, \{ drink, eat\} }}\\
    Oczekiwana odpowiedź: \texttt{TRUE}\\
    Odpowiedź programu: \texttt{TRUE}\\
    Odpowiedż jest poprawna, ścieżkę można zobaczyć na poniższym obrazku.
    \begin{figure}[H]
    \centering
    \includegraphics[scale=0.5]{test3_5}
    \end{figure}
    
    
    %\item {\large\texttt{?}}\\
    %Oczekiwana odpowiedź: \texttt{}\\
    %Odpowiedź programu: \texttt{}\\
    %Odpowiedż jest poprawna, ponieważ .
\end{itemize}

\newpage
\subsection{Test 4}
\newpage
\subsection{Test 5}
\newpage
\subsection{Test 6}
\newpage
\subsection{Test 7}
\newpage
\subsection{Test 8}
\newpage
\subsection{Test 9}
\newpage
\section{Wnioski}

\end{document}
