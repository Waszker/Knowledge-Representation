\documentclass{article}
\usepackage{float}
\usepackage[polish]{babel}
\usepackage[utf8]{inputenc}
\usepackage{polski}
\usepackage{algpseudocode}
\usepackage{algorithm}
\usepackage{hyperref}
\frenchspacing
\setcounter{tocdepth}{2}
\usepackage{graphicx}
\graphicspath{ {images/} }

%%% fix for \lll
\let\babellll\lll
\let\lll\relax

\usepackage{amssymb}

%%% fix for \lll
\let\mathlll\lll
\let\lll\babellll


\begin{document}
\begin{titlepage}
\newcommand{\HRule}{\rule{\linewidth}{0.5mm}}
\newcommand{\Action}[1]{\textsc{#1}}
\center
%----------------------------------------------------------------------------------------
\textsc{\LARGE Politechnika Warszawska}\\[0.3cm]
\textsc{\Large Wydział Matematyki i Nauk Informacyjnych}\\[0.6cm]
% logo
\includegraphics[width=2cm, height=2cm]{logo}\\[0.6cm]
\textsc{\Huge Reprezentacja wiedzy}\\[0.3cm]
%---------------------------------------------------------------------------------------
\HRule \\[0.4cm]
{ \LARGE \bfseries Programy działań z efektami domyślnymi}\\[0.1cm]
 
%----------------------------------------------------------------------------------------
\HRule \\[0.4cm]
{  \bfseries Raport końcowy}\\[1.2cm]
% authors
\begin{flushright}
\Large \emph{Autorzy:}\\[0.5cm]
Dragan Łukasz\\
Flis Mateusz\\
Izert Piotr\\
Pielat Mateusz\\
Rząd Przemysław\\
Siry Roman\\
\textbf{Waszkiewicz Piotr}\\
Zawadzka Anna\\[0.9cm]
\end{flushright}
% date
\vfill
{\large 14 czerwca 2016}\\[1cm]
\end{titlepage}
\newpage

\tableofcontents
\newpage

\section{Opis zadania}

Zadaniem projektu jest opracowanie i zaimplementowanie języka akcji dla specyfikacji podanej klasy systemów dynamicznych oraz odpowiadający mu język kwerend.\\

System dynamiczny spełnia podane założenia:
\begin{enumerate}
\item Prawo inercji
\item Niedeterminizm i sekwencyjność działań
\item Pełna informacja o wszystkich akcjach i wszystkich ich skutkach bezpośrednich
\item Z każdą akcją związany jest:
\begin{enumerate}
\item Warunek początkowy (ew. true)
\item Efekt akcji
\item Jej wykonawca
\end{enumerate}
\item Skutki akcji:
\begin{enumerate}
\item Pewne (zawsze występują po zakończeniu akcji)
\item Domyślne (preferowane. Zachodzą po zakończeniu akcji, o ile nie jest wiadomym, że nie występują)
\end{enumerate}
\item Efekty akcji zależą od jej stanu, w którym akcja się zaczyna i wykonawcy tej akcji
\item W pewnych stanach akcje mogą być niewykonalne przez pewnych (wszystkich) wykonawców
\end{enumerate}

Programem działań nazywać będziemy ciąg $((A_{1},w_{1}), (A_{2},w_{2}), …, (A_{n},w_{n}))$, 
gdzie $A_{i}$ jest akcją, zaś $w_{i}$ jej wykonawcą, $n = 0, 1, 2, \dots$ gdzie $w_{i} = \epsilon$ oznacza dowolnego wykonawcę.\\


Język kwerend zapewnia uzyskanie odpowiedzi na następujące pytania:
\begin{enumerate}
\item Czy podany program działań jest wykonywalny zawsze/kiedykolwiek?
\item Czy wykonanie podanego programu działań z dowolnego stanu spełniającego warunek $\pi$ prowadzi zawsze/kiedykolwiek/na ogół do stanu spełniającego warunek celu $\gamma$ ?
\item Czy z dowolnego stanu spełniającego warunek $\pi$ cel $\gamma$ jest osiągalny zawsze/kiedykolwiek/na ogół?
\item Czy wskazany wykonawca jest zaangażowany w realizację programu zawsze/kiedykolwiek?
\end{enumerate}

\section{Opis klas}
Przygotowany program, realizujący postawiony cel, został napisany w języku programowania C\# i wykonany w technologii Windows Forms. Składa się on z szeregu klas będących odzwierciedleniem pojęć występujących w teorii reprezentacji wiedzy, mających na celu łatwiejsze napisanie i zrozumienie sposobu działania programu. 

\subsection{Klasy fluentów, aktorów i akcji}
Klasy te służą do identyfikowania poszczególnych elementów występujących w dziedzinie wprowadzonej przez użytkownika. Rozróżnialne są za pomocą unikalnych nazw nadawanych im w trakcie działania programu.

\subsection{Klasy zdań logicznych}
W tym zbiorze znajdują się klasy, które są realizacją podstawowych operatorów z logiki klasycznej:
\begin{itemize}
    \item Negacja
    \item Koniunkcja
    \item Alternatywa
    \item Implikacja
    \item Równoważność
\end{itemize}

\subsection{Klasy zdań}
W ramach przygotowanego programu zostały zrealizowane klasy dla każdego z typów zdań możliwych do zdefiniowania:
\begin{itemize}
\item {\large\texttt{initially $\alpha$}}\\
Zawiera formułę $\alpha$ zdefiniowaną za pomocą klas zdań logicznych.
\item {\large\texttt{$(A,W)$ causes $\alpha$ if $\pi$}}\\
Zawiera akcję, listę wykonawców, informację o wykluczeniu wykonawców typu logicznego oraz formuły $\alpha$ i $\pi$ zdefiniowane za pomocą klas zdań logicznych. Warunek $\pi$ może być pominięty, co oznacza, że jest zawsze prawdziwy, wtedy powyższe zdanie jest postaci {\large\texttt{$(A,W)$ causes $\alpha$}}.
\item {\large\texttt{$(A,W)$ typically causes $\alpha$ if $\pi$}}\\
Klasa ta zbudowana jest analogicznie do klasy powyżej, lecz reprezentuje ona zdanie, którego efekt jest typowy.
\item {\large\texttt{always $\alpha$}}\\
Zawiera formułę $\alpha$ zdefiniowaną za pomocą klas zdań logicznych.
\item {\large\texttt{impossible $(A,W)$ if $\pi$}}\\
Zawiera akcję, listę wykonawców, informację o wykluczeniu wykonawców typu logicznego oraz warunek $\pi$ zdefiniowany za pomocą klas zdań logicznych. 
\item {\large\texttt{$(A,W)$ releases $f$ if $\pi$}}\\
Zawiera akcję, listę wykonawców, informację o wykluczeniu wykonawców typu logicznego, fluent oraz warunek $\pi$ zdefiniowany za pomocą klas zdań logicznych. 
\item {\large\texttt{$(A,W)$ preserves $f$ if $\pi$}}\\
Zawiera akcję, listę wykonawców, informację o wykluczeniu wykonawców typu logicznego, fluent oraz warunek $\pi$ zdefiniowany za pomocą klas zdań logicznych.
\end{itemize}

\subsection{Klasa dziedziny}
Klasa ta jest programową reprezentacją domeny wprowadzanej przez użytkownika. Zawiera ona listy zdań wszystkich typów, dopuszczalnych w ramach programu. Ponieważ wraz z uzupełnianiem danych liczba i rodzaj zdań może się zmieniać, oferuje ona niezbędne metody służące do ich zmiany, usuwania i modyfikowania.

\subsection{Klasa stanu}
Ponieważ w trakcie działania programu istnieje potrzeba rozróżniania możliwych stanów opisywanego systemu, powstała klasa odpowiadająca takiemu pojedynczemu stanowi. Każda taka klasa zawiera listę wartościowań dla każdego fluentu występującego w dziedzinie i jest ich konstruowanych w programie tyle, ile występuje unikalnych wartościowań ($2^{|fluent set|}$).

\subsection{Klasa Graph}
Reprezentuje graf zależności między poszczególnymi stanami.

\subsection{Klasa Edge}
Każda ścieżka w konstruowanym grafie łączy dwa wierzchołki reprezentujące poszczególne stany między którymi istnieje połączenie opisane w dziedzinie - akcja wykonywana przez pewny zbiór wykonawców. Ponieważ akcje mogą mieć skutki typowe i nietypowe, wyróżnia się także rodzaj krawędzi. Klasa zawiera dwa stany (wierzchołki), akcję i jej wykonawcę oraz informację o nietypowości efektu akcji jako zmienną logiczną.

\subsection{Klasa World}
Jej składowymi są zbiory fluentów, akcji, aktorów oraz dziedzina. Klasa ta odpowiedzialna jest za budowanie struktury $S=(\Sigma, \sigma_{0}, ResAb, ResN)$, gdzie:
\begin{itemize}
\item $\Sigma$ - zbiór stanów
\item $\sigma_{0} \in \Sigma$ - stan początkowy
\item $ResAb, ResN$ : $A\times V \times \Sigma \to 2^{\Sigma}$ są funkcjami przejść. $ResAb$ jest funkcją przejść nietypowych, $ResN$ jest funkcją przejść typowych.
\end{itemize}
Klasa ta dostarcza także metodę budowania grafu zależności na podstawie skonstruowanej struktury $S$. 

\subsection{Klasa kroku scenariusza}
Krok scenariusza zdefiniowany jest poprzez akcję i wykonawcę tejże akcji.

\subsection{Klasa scenariusza}
Składa się z listy kroków scenariusza.

\subsection{Klasy kwerend}
Każda kwerenda, która może zostać zdefiniowana w programie, reprezentowana jest przez osobną klasę. Wyróżniamy następujące rodzaje kwerend:
\begin{itemize}
\item {\large\texttt{always/ever executable $Scenario$}} \\
Klasy reprezentujące te kwerendy (odpowiednio $always$ i $ever$) przechowują obiekt klasy Scenariusza.
\item {\large\texttt{always/ever/typically accessible $\gamma$ if $\pi$}}\\
Klasy reprezentujące te kwerendy (odpowiednio $always$, $ever$ i $typically$) przechowują formuły $\gamma$ i $\pi$ zdefiniowane za pomocą klas zdań logicznych.
\item {\large\texttt{always/ever/typically accessible $\gamma$ if $\pi$ when $Scenario$}}\\
Klasy te zbudowane są analogicznie jak klasy zdefiniowane powyżej, z tym, że dodatkowo przechowują obiekt klasy Scenariusza.
\item {\large\texttt{always/ever partakes $w$ when $Scenario$}}\\
Klasy reprezentujące te kwerendy (odpowiednio $always$ i $ever$) przechowują wykonawcę oraz obiekt klasy Scenariusza.
\end{itemize}
Każda klasa reprezentująca kwerendę dziedziczy po klasie Query i implementuje metodę Evaluate, która daje odpowiedź na zadane pytanie na podstawie zbudowanego wcześniej grafu zależności.
\newpage

\section{Algorytmy}

\subsection{Wyznaczanie zbioru stanów}
Algorytm na podstawie zbioru fluentów wyznacza zbiór wszystkich stanów. Oparty jest na idei algorytmu z powrotami, który wyznacza wszystkie kombinacje n-elementowego ciągu składającego się z 0 i 1.

\begin{algorithm}[H]
\begin{algorithmic}
\State $B \gets $ pusty stos
\State $n \gets $ liczba fluentów
\State $k \gets $ 0
\State $S \gets $ lista stanów
\Function{Backtrack}{B, n, k}
	\If{$k = n$}
    	\Call{append}{$S, B$}
    \EndIf
	\State \Call{put}{$B, 1$}
	\State \Call{BackTrack}{$B, n, k + 1$}
	\State \Call{pop}{$B$}
	\State \Call{put}{$B, 0$}
	\State \Call{BackTrack}{$B, n, k + 1$}
	\State \Call{pop}{$B$}
	\State \Return{$S$}
\EndFunction
\end{algorithmic}
\end{algorithm}

\newpage
\subsection{Wyznaczanie stanu początkowego}

Algorytm wyznaczania stanu początkowego polega na wygenerowaniu wszystkich możliwych kombinacji wszystkich fluentów danego świata oraz \textit{przefiltrowaniu} ich przez zdania \texttt{always}. Otrzymane w ten sposób dozwolone stany świata są następnie kolejno sprawdzane pod kątem zgodności ze zdaniami \texttt{initially}. W poprawnie opisanym świecie powinien istnieć tylko jeden stan który nie zostanie odrzucony w trakcie takiego procesu.

\begin{algorithm}[H]
\begin{algorithmic}
\State $H \gets $ zbiór wszystkich kombinacji fluentów $\mathcal{F}$
\ForAll{ $\texttt{always} \ \alpha \in \mathfrak{D}$}
	\ForAll{$\sigma \in H$}
		\If{$\sigma \nvDash \alpha$}
			\State $H \gets H \setminus \{\sigma\}$
		\EndIf
	\EndFor
\EndFor
\ForAll{$\texttt{initially} \ \alpha \in \mathfrak{D}$}
	\ForAll{$\sigma \in H$}
		\If{$\sigma \nvDash \alpha$}
			\State $H \gets H \setminus \{\sigma\}$
		\EndIf
	\EndFor
\EndFor
\end{algorithmic}
\end{algorithm}

\newpage
\subsection{Obliczanie zbiorów ResN i ResAb}

Algorytm obliczania zbiorów $ResN$ i $ResAb$ oraz zbiorów pośrednich $Res_0$, $Res^{-}$, $Res^{+}_{0}$ działa na analogicznej zasadzie \textit{filtrowania} zbioru stanów zdaniami w algorytmie wyznaczania stanu początkowego. Na wyższym poziomie abstrakcji pseudokod wygląda następująco:\footnote{Dla uproszczenia pseudokodu przez $Res$ i $New$ rozumie się kolejno $Res(A,Akt,\sigma)$ oraz $New(A,Akt,\sigma_0,\sigma_1)$}

\begin{algorithm}[H]
\begin{algorithmic}
\State $\Sigma \gets $ zbiór wszystkich kombinacji fluentów $\mathcal{F}$
\State $\Sigma \gets$ stany $\sigma \in \Sigma$ zgodne ze zdaniami $\texttt{always}$.
\State $Res_0 \gets \Sigma$
\If{dowolne zdanie $\texttt{impossible}$ blokuje $A$ dla $Akt$ i $\sigma$}
	\State $Res_0 \gets \emptyset$ 
\EndIf
\State $Res_0 \gets$ stany $\sigma \in Res_0$ zgodne ze zdaniami $\texttt{causes}$ dla $A$ i $Akt$
\State $Res_0 \gets$ stany $\sigma \in Res_0$ zgodne ze zdaniami $\texttt{preserves}$ dla $A$ i $Akt$
\State $Res^- \gets$ stany $\sigma \in Res_0$ o minimalnych zbiorach $New$
\State $Res^+_0 \gets$ stany $\sigma \in Res^+_0$ zgodne ze zdaniami $\texttt{typically causes}$ dla $A$ i $Akt$
\State $ResN \gets$ stany $\sigma \in Res^+_0$ o minimalnych zbiorach $New$
\State $ResAb \gets Res^- \setminus ResN$ 
\end{algorithmic}
\end{algorithm}

\newpage
\subsection{Konstrukcja grafu zależności}

Graf przejść między stanami generowany jest na podstawie obliczonych zbiorów $ResN$ i $ResAb$. Pseudokod algorytmu wygląda następująco:

\begin{algorithm}[H]
\begin{algorithmic}
\State $States \gets $ zbiór wszystkich stanów
\State $Actions \gets $ zbiór wszystkich akcji
\State $Actors \gets $ zbiór wszystkich aktorów
\State $E \gets \emptyset$
\ForAll{$\texttt{state} \in States$}
	\ForAll{$\texttt{action} \in Actions$}
		\ForAll{$\texttt{actor} \in Actors$}
			\ForAll{$\texttt{target} \in ResAb_{action, actor, state}$}
				\State $e \gets$ nowa krawędź
				\State $e.From \gets \texttt{state}$
				\State $e.To \gets \texttt{target}$
				\State $e.Action \gets \texttt{action}$
				\State $e.Actor \gets \texttt{actor}$
				\State $e.Abnormal \gets \texttt{True}$
				\State $E \gets E \cup \{e\}$
			\EndFor
			\ForAll{$\texttt{target} \in ResN_{action, actor, state}$}
				\State $e \gets$ nowa krawędź
				\State $e.From \gets \texttt{state}$
				\State $e.To \gets \texttt{target}$
				\State $e.Action \gets \texttt{action}$
				\State $e.Actor \gets \texttt{actor}$
				\State $e.Abnormal \gets \texttt{False}$
				\State $E \gets E \cup \{e\}$
			\EndFor
		\EndFor
	\EndFor
\EndFor
\State $G \gets \left(States, E\right)$
\end{algorithmic}
\end{algorithm}
\newpage

\subsection{Obliczanie odpowiedzi na zadaną kwerendę}

\subsubsection{Kwerendy \texttt{always/ever executable Scenario}}

Kwerendy \texttt{always/ever executable Scenario} sprawdzają, czy dany scenariusz jest wykonywalny zawsze/kiedykolwiek z dowolnego stanu. Pseudokod algorytmu wygląda następująco:

\begin{algorithm}[H]
\begin{algorithmic}
\State $Scenario \gets $ zadany scenariusz
\State $States \gets $ zbiór wszystkich dostępnych stanów
\ForAll{$\texttt{step} \in Scenario$}
    \State $NewStates \gets \emptyset$
    \ForAll{$\texttt{state} \in States$}
        \State $Edges \gets $ wszystkie krawędzie dostępne w danym stanie dla kroku \texttt{step}
        \If{Wersja ALWAYS i $Edges.count() == 0$}
        	\State return \texttt{FALSE} - ścieżka niewykonywalna - brak krawędzi
        \EndIf
        \ForAll{$\texttt{edge} \in Edges$}
            \State $NewStates.add(edge.to)$
        \EndFor
    \EndFor
    \If{Wersja EVER i $NewStates.count() == 0$}
        \State return \texttt{FALSE} - nie było krawędzi z żadnego stanu - żadna ścieżka nie jest wykonywalna
    \EndIf
    \State $States \gets NewStates$
\EndFor
\State return \texttt{TRUE}

\end{algorithmic}
\end{algorithm}

\newpage
\subsubsection{Kwerendy \texttt{always/ever Actor partakes when Scenario}}

Kwerendy \texttt{always/ever Actor partakes when Scenario} sprawdzają, czy dany aktor zawsze/kiedykolwiek bierze udział w wykonywaniu podanego scenariusza. Scenariusz może być wykonany z dowolnego niesprzecznego stanu początkowego. Pseudokod algorytmu wygląda następująco:

\begin{algorithm}[H]
\begin{algorithmic}

\State $Actor \gets $ wybrany aktor
\State $Scenario \gets $ zadany scenariusz
\State $States \gets $ zbiór wszystkich dostępnych stanów
\ForAll{$\texttt{step} \in Scenario$}
    \State $NewStates \gets \emptyset$
    \ForAll{$\texttt{state} \in States$}
        \State $Edges \gets $ wszystkie krawędzie w danym stanie dla kroku \texttt{step}
        \ForAll{$\texttt{edge} \in Edges$}
            \If{$edge.Actor == Actor$}
                \If{Wersja EVER}
        	        \State return \texttt{TRUE} - dany aktor bierze udział
        	    \EndIf
        	\Else - inny aktor
        	    \State $NewStates.add(edge.to)$ - dalsze przeszukanie ścieżki
            \EndIf
        \EndFor
    \EndFor
    \If{Wersja ALWAYS i $NewStates.count() == 0$}
        \State return \texttt{TRUE} - nie dodano żadnego nowego stanu, czyli na wszystkich ścieżkach znalazł sie już szukany aktor
    \EndIf
    \State $States \gets NewStates$
\EndFor
\State return \texttt{FALSE}

\end{algorithmic}
\end{algorithm}
\newpage

\subsubsection{Kwerendy \texttt{always/ever/typically accessible $\gamma$ if $\pi$ when Scenario} }

Kwerendy \texttt{always/ever/typically accessible $\gamma$ if $\pi$ when Scenario} sprawdzają, czy wykonanie podanego scenariusza z każdego stanu spełniającego warunek $\pi$ zawsze/kiedykolwiek/na ogół doprowadzi do stanu spełniającego warunek $\gamma$. Pseudokod algorytmu wygląda następująco:

\begin{algorithm}[H]
\begin{algorithmic}
\State $StatesPi \gets $ zbiór wszystkich stanów spełniających warunek $\pi$ 
\State $scenario \gets $ scenariusz kwerendy
\ForAll{$\texttt{startState} \in StatesPi$}
    \State $states \gets \{startState\}$
	\ForAll{$\texttt{step} \in scenario$}
	    \State $nextS \gets $ zbiór stanów
		\ForAll{$\texttt{state} \in states$}
			\State $nextS \gets nextS \cup ($stany z krawędzią z $state$ z krokiem $step)$
		\EndFor 
		\State $states \gets nextS$
	\EndFor
	\If{wersja $always$ i $states$ zawiera stan nie spełniający $\gamma$} 
	    \State \Return{$FALSE$}
	\EndIf
	\If{wersja $ever$ i $states$ nie zawiera stanu spełniającego $\gamma$} \State \Return{$FALSE$}
	\EndIf
\EndFor
\State \Return{$TRUE$}
\end{algorithmic}
\end{algorithm}
\newpage

\subsubsection{Kwerendy \texttt{always/ever/typically accessible $\gamma$ if $\pi$} }

%TUTAJ OPIS!!!!






\newpage
\section{Moduły dodatkowe}
\subsection{Biblioteka Microsoft Automatic Graph Layout}
Biblioteka ta została użyta do wizualizacji grafu zależności. Dostępna jest jako oprogramowanie open source pod adresem: \\ \url{https://github.com/Microsoft/automatic-graph-layout.git}. \\\\
Pakiet złożony jest z trzech bibliotek:
\begin{itemize}
    \item Layout engine (Microsoft.MSAGL.dll)
    \item Drawing module (Microsoft.MSAGL.Drawing.dll)
    \item Viewer control (Microsoft.MSAGL.GraphViewerGDIGraph.dll)
\end{itemize}
Podstawowe funkcjonalności biblioteki:
\begin{itemize}
    \item Przybliżanie, oddalanie i przesuwanie grafu
    \item Dowolna konfiguracja krawędzi i wierzchołków
\end{itemize}

\newpage
\section{Podział prac}
\begin{itemize}
    \item Dragan Łukasz - wyznaczanie zbioru stanów, ewaluacja formuł, testy
    \item Flis Mateusz - wizualizacja grafu zależności, testy
    \item Izert Piotr - implementacja kwerend \texttt{executable} i \texttt{partakes}, testy
    \item Pielat Mateusz - wyznaczanie zbiorów $Res$, testy
    \item Rząd Przemysław - konstrukcja grafu zależności, testy
    \item Siry Roman - implementacja kwerend \texttt{accessible}, testy 
	\item Waszkiewicz Piotr - podstawowe klasy, graficzny interfejs użytkownika, testy
    \item Zawadzka Anna - podstawowe klasy, graficzny interfejs użytkownika, testy
\end{itemize}
\end{document}